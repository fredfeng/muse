%%%%%%%%%%%%%%%%%%%%%%%%%%%%%%%%%%%%%%%%%%%%%%%%%%%%%%%%%%%%%%%%%%%%%%%%%%%%%%
  \headerbox{Objective}{name=problem,column=0,row=0}{
%%%%%%%%%%%%%%%%%%%%%%%%%%%%%%%%%%%%%%%%%%%%%%%%%%%%%%%%%%%%%%%%%%%%%%%%%%%%%%
   \noindent{\centering\includegraphics[scale=0.9]{images/sypet.png}\\}
   \vspace{0.5mm}
 {\sc SyPet} is a component-based synthesizer for {\it large} libraries that automatically 
 synthesizes executable programs by composing API calls.
 
 \vspace{3.6mm}
 Key components:
 \begin{enumerate}
 \item {\bf \emph{Synthesis of program sketches:}} {\sc SyPet} uses Petri nets to generate programs sketches from signatures of the desired method 
 and underlying library components. 
 \item {\bf \emph{Completion of program sketches:}} {\sc SyPet} generates constraints on the synthesized 
 program with holes and uses a SAT solver to find a candidate method.
 \end{enumerate}
   \vspace{0.3em}
 }


